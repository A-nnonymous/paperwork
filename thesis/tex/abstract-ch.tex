%-------------------------------------------------
% FileName: abstract-ch.tex
% Author: Safin (zhaoqid@zsc.edu.cn)
% Version: 0.1
% Date: 2020-05-12
% Description: 中文摘要
% Others: 
% History: origin
%------------------------------------------------- 

% 以下不用改动-------------------------------------
% 断页
\clearpage
% 页码从1开始计数
\setcounter{page}{1}
% 大写罗马数字显示页码
\pagenumbering{Roman}
% 加入书签, bm@abstractname要唯一
\currentpdfbookmark{\defabstractname}{bm@abstractname}
% \chapter*{} 表示不编号,不生成目录
% \markboth{}{} 用于页眉
% 此处以中文题目作为章题目
\chapter*{\deftitle\markboth{\defabstractname}{}}


% 修改摘要和关键词---------------------------------
% 中文摘要
\abstract{
随着新冠肺炎在全球的传播,在全球化的今天,每一个国家都不可能独善其身,在一国之内、国家之间对有限医疗、行政资源的分配和利用,是在疫情环境下人类作为一个整体获得最大利益所必须考虑的问题。对疫情的准确预测在政策执行、医疗调度上有着很大的作用,而对疫情态势本身的理解则能提供更高的参考意义,对于小样本时序数据,现有的机器学习算法在精度和可解释性上不可兼得,且容易造成过拟合的结果,本文采用了一种新型的拟合、解释方法,为疫情态势分析与解释提供了一种新的可行方法。

本文采用了一种前沿的模式识别算法,用于对美国本土疫情态势的拟合与预测,并利用其的可解释性对其训练完毕后的模型中拟合的命题公式进行相关的态势分析,最终为医疗资源分配和疫情防控政策提供有价值的参考信息。
%经过...测试,实现....
}

% 中文关键词
% 关键词是供检索用的主题词条,应采用能覆盖毕业设计(论文)主要内容的通用技术词条(参照相应的技术术语标准)。关键词一般为3~5个,每个关键词不超过5个字。

\keywords{Tsetlin机;新冠肺炎;疫情态势分析;机器学习可解释性}


